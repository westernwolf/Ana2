\section{Uneigentliche Integrale}
Das Intervall ($\vec{x}$) kann $\infty$ sein, oder der Wertebereich ($\vec{y}$).
\begin{align*}
&\int_a^\infty f(x)\dx:=\lim_{t\rightarrow\infty}\left(\int_a^t f(x)\dx\right) = 
	\begin{cases} 
		 konvergent & (endlicher Flaecheninhalt)\\
		divergent
	\end{cases}
	\\
	&\int_{-\infty}^{a}f(x)\dx = lim_{t\rightarrow-\infty}\left(\int_t^af(x)\dx\right)\\
	&\rightarrow\qquad F(\infty)-F(a);F(a)-F(-\infty) \rightarrow \text{konventionelle Limit-Rechnung}\\
	&\int^\infty_{-\infty}f(x)\dx\stackrel{\text{Zerlegung}}{=}\left(\int^a_{-\infty}f(x)\dx\right)\left(\int^\infty_af(x)\dx\right) \color{red}{(! \D \stackrel{!}{=}\R)}
\end{align*}
\subsection{Folgerungen und Spezialfälle}
\begin{align*}
&\int^\infty_A \frac{1}{x^\alpha}\dx \text{ konvergiert} \Leftrightarrow\alpha>1\quad(\forall A>0)\\
&\int^\infty_af(x)\dx \text{ konvergiert} \Rightarrow f(x)=y< \stackrel{x\rightarrow\infty}{\rightarrow}0 \\
&\int^\infty_af(x)\dx \text{ divergiert} \Leftarrow f(x)=y< \stackrel{x\rightarrow\infty}{\rightarrow}\R\backslash \{0\} 
\end{align*}
\subsection{Majorante/Minorante}
Dies ist Vergleichbar mit Majorante und Minorante einer Folge.\\
Man wählt eine Funktion die immer grösser ist als der Betrag der Funktion, von welcher man bestimmen möchte, ob sie konvergiert ({\color{red}p. 503}). Wenn die Majorante konvergiert, konvergiert auch die Funktion.
\begin{align*}
& x \in\left[a;\infty\right),\abs{f(x)} \leq g(x):\\
&\int^\infty_a g(x)\dx \text{ konvergiert}&\Rightarrow\int^\infty_a \abs{f(x)}\dx \text{ konvergiert}\\
&&\stackrel{{\color{red}\xcancel{\Leftarrow}}}{\Rightarrow}\int^\infty_af(x)\dx \text{ konvergiert}
\end{align*}
Wenn die Minorante divergiert, divergiert auch die Funktion.
\begin{align*}
&0\stackrel{!}{\leq}g(x)\leq f(x):\\
&\int^\infty_a g(x)\dx=\infty \Rightarrow\int^\infty_af(x)\dx =\infty
\end{align*}

\subsubsection{Standard-Abschätung}
\begin{align*}
\abs{\sin({\dots)}}&\leq1\\
\abs{\cos({\dots)}}&\leq1\\
\end{align*}

\subsection{Polstellen u.ä.}
Z.i.g. $b$ ist z.B. eine Polstelle. $b^-$ ist kleiner als $b$.
\begin{align}
x\in\left[a,b\right):\quad \int^{b^-}_a f(x)\dx:=\lim_{t\rightarrow b^-} \int^{t<b}_af(x)\dx\\
x\in\left(a,b\right]:\int^b_{a^+}f(x)\dx:= \lim_{t\rightarrow a^+}\left(\int^b_tf(x)\dx\right)
\end{align}

\subsection{Cauchy Hauptwert, aka. Polstelle in $\D$}
Abgekürzt C.H. (oder P.V. für prinziple Value). Wenn im zu integrierenden Bereich (ohne Rand) eine
Polstelle liegt, kann man es aufteilen. Im schlimmsten Fall erhält man $\dots \infty -\infty$, da die Integrale einzeln ausgewertet werden müssen. Dies ergäbe eine unbestimmbare Lösung.  Dies wird mit C.H. umgangen.
\begin{align*}
\int^b_a f(x)\dx=\lim_{\epsilon\rightarrow 0^+}\left(\int^{x_0-{\color{orange}\epsilon}}_a f(x)\dx +\int^b_ {x_0+{\color{orange}\epsilon}}f(x)\dx \right)
\end{align*}
Durch das $\epsilon$ sind die Integrationsgrenzen gleich weit von der Polstelle entfernt.\\
{\color{red}Lücken/Unstetigkeiten/Polstellen dürfen nicht ignoriert werden $\rightarrow$ Zerlegung des Intervall}
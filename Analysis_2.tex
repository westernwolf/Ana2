% !TEX TS-program = pdflatex
% !TeX encoding = UTF-8
% !TeX spellcheck = de_CH

\documentclass[a4paper,10pt]{scrartcl}
\usepackage[utf8]{inputenc}
\usepackage[T1]{fontenc}
\usepackage[ngerman]{babel}
\usepackage{amsmath}
\usepackage{amssymb}
\usepackage{graphicx}
\usepackage{color}
	\definecolor{orange}{RGB}{255,127,0}

\title{Analysis (Spick)}
\author{Westernwolf}

\begin{document}

\maketitle
\newpage

\tableofcontents
\newpage

\section{bla die bla Resume}
	%\input{AnalysisRes}
\section{Diskrete Mathematik}%%%%%%%%%%%%%%%%%%%%%%%%%%%%%%%%%%%%%%%%%%%%%%%%%%%%%%%%%
	
	\subsection{Beweise}
		\subsubsection{Direkter Beweis}
		
		\subsubsection{Indirekter Beweis}
		
		\subsubsection{Induktionsbeweis}
		
		\subsubsection{Beweis durch Wiederspruch}

\section{Beweise durch völlständige Induktion}
	
	\subsection{Prinzip der vollständigen Induktion}u
	
\section{Folgen}
	
\section{Reihen (series)}
	
	\subsection{Spezielle Reihen}
		
	\subsection{Summen von Reihen}
		
	\subsection{KonvergenzKriterien}
		
		\subsubsection{Mittels der Definition}
		$\sqrt{•}$
			
		\subsubsection{$\lim_{n\to\infty}a_n=0$}
			
		\subsubsection{Majoranten- und Minorantenkriterium}
			
		\subsubsection{Vergleichskriterium}
			
		\subsubsection{Quotientenkriterium}
			
		\subsubsection{Wurzelkriterium}
			
		\subsubsection{Integralkriterium}
			
		\subsubsection{Leibnitz-Kritrerium}
			
		\subsubsection{Absolute Konvergenz}
			
		\subsubsection{Cauchy-Kondensationstest}
			
\end{document}

